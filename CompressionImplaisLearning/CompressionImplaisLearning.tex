\documentclass[../main.tex]{subfiles}

\newcommand*{\No}{No.}
\begin{document}

\subsection{PAC-learning}

\begin{definition}
Класс гипотез~$\mathcal{H}$ является PAC-обучаемым над множеством объектов~$Z=z_1, \cdots z_d$ для функции потерь~$l:\mathcal{H}\times Z\to \mathbb{R}_{+},$ если существует функция~$m_{\mathcal{H}}:\left(0,1\right)^2\to \mathbb{N}$ и алгоритм~$\mathcal{A}$ со свойством:
для всех $\varepsilon, \delta \in \left(0,1\right)$, для любого распределения~$\mathcal{D}$ над множеством объектов~$Z$ алгоритм~$\mathcal{A}$ возвращает такое $h\in \mathcal{H},$ что с вероятностью~$1-\delta$ выполняется:
\[
 \label{def:1}
\begin{aligned}
\mathsf{E}_{z\sim \mathcal{D}}\left[l\bigr(h, z\bigr)\right] \leq \min_{h'\in \mathcal{H}} \mathsf{E}_{z\sim \mathcal{D}}\left[l\bigr(h', z\bigr)\right] +\varepsilon.
\end{aligned}
\]
\end{definition}
Пример для задачи классификации выражение \eqref{def:1} можно переписать в виде:
\[
\begin{aligned}
\begin{aligned}
P\bigr[p\bigr(\left\{h\bigr(z\bigr)\not=f\bigr(z\bigr) \right\}\bigr)>\varepsilon\bigr] \leq \delta
\end{aligned}
\end{aligned}
\]

\subsection{Sample Compression scheme}
Схема сжатия данных с параметром $k$ состоит из двух отображений~$\left(\kappa, \rho\right):$
\begin{enumerate}
    \item $\kappa$ получает на вход выборку~$S,$ а на выходе получаем пару~$\left(S', I\right),$ где $|S'|=k$;
    \item $\rho$ получает на вход пару~$\left(S', I\right)$ на выходе выдает гипотезу~$h$.
\end{enumerate}
Причем выполняется следующее условие:
\begin{enumerate}
    \item $\kappa\bigr(Y, y\bigr) = \left(\left(Z, z\right), I\right)$;
    \item $\rho\bigr(\kappa\bigr(Y, y\bigr)\bigr)|_{Y} = y$.
\end{enumerate}

\subsection{Compression implais learning}
Любую схему сжатия с параметром $k$ можно рассматривать как алгоритм обучения~$A = \rho \circ \kappa$. То что данный алгоритм PAC-обучаем доказывает следующая теорема.
\begin{theorem}
Алгоритм обучения~$A = \rho \circ \kappa$ является PAC-обучаемым, то есть
\[
\begin{aligned}
P\bigr[p\bigr(\left\{h\bigr(z\bigr)\not=f\bigr(z\bigr) \right\}\bigr)>\varepsilon\bigr] \leq |I|\sum_{j=1}^{k}{{d}\choose{j}}\left(1-\varepsilon\right)^{m-j},
\end{aligned}
\]
где $p$ распределение над $Z$.
\end{theorem}
\begin{proof}
Сначала заметим, что всего существует
\[
\begin{aligned}
\sum_{j=1}^{k}{{d}\choose{j}}
\end{aligned}
\]
подмножеств $T$ множества $Z$ размера не более k. С другой стороны всего есть $|I|$ вариантов выбрать информацию сжатия  $i \in I$.
Из выше описанного получаем, что каждой паре $\left(T, i\right)$ соответствует своя функция 
\[
\begin{aligned}
h_{T,i}=\rho\bigr(\left(T, i\right), i\bigr).
\end{aligned}
\]
С построения $h_{T, i}$ следует, что $h_{T, i}$ не зависит от $Z \setminus T$, тогда получаем, что если для одного объекта $z\in Z \setminus T$ вероятность ошибки:
\[
        p\bigr(\left\{h_{T,i}\bigr(z\bigr) \not=f(z)\right\}\bigr) \geq \varepsilon,
\]
то для всех объектов $z\in Z \setminus T$ вероятность того, что на нем нет ошибки равна
\[
    \label{eq:2}
    \begin{aligned}
        \prod_{z\in Z \setminus T}p\bigr(\left\{h_{T,i}\bigr(z\bigr) =f(z)\right\}\bigr) \leq \left(1-\varepsilon\right)^{m-|T|}.
    \end{aligned}
\]
Получаем, что для любого $h_{T, i}$ выполняется неравенство \eqref{eq:2}.

И того получаем, что для произвольной $h_{T, i}$ выполняется неравенство:
\[
\begin{aligned}
    P\bigr[p\bigr(\left\{h_{T, i}\bigr(z\bigr)\not=f\bigr(z\bigr) \right\}\bigr)>\varepsilon\bigr] \leq \left(1-\varepsilon\right)^{m-|T|},
\end{aligned}
\]

Рассмотрим множество функций при фиксированном $i \in I$:
\[
    \label{eq:3}
    \begin{aligned}
        \mathcal{H}_{i} = \left\{h_{T, i}: |T| \leq k\right\},
    \end{aligned}
\]
тогда для алгоритма $A$ для подмножества функций, которые получены при помощи сжатой информации $i$ получаем:
\[
\begin{aligned}
    \label{eq:4}
    P\bigr[p\bigr(\left\{h_{T, i}\bigr(z\bigr)\not=f\bigr(z\bigr) \right\}\bigr)>\varepsilon\bigr] \leq \sum_{j=1}^{k}{{d}\choose{j}}\left(1-\varepsilon\right)^{m-j},
\end{aligned}
\]
где $h_{T, i}$ это лучший алгоритм из множества $\mathcal{H}_{i}$.

Теперь заметим, что финальная функция $h$ принадлежит множеству:
\[
    \label{eq:5}
    \begin{aligned}
        \mathcal{H}_{\kappa, \rho} = \left\{h_{T, i}: |T| \leq k, i \in I\right\}.
    \end{aligned}
\]
Вспомним, что для каждого $T$ таких функций $|I|,$ из чего уже для произвольного $h$ используя выражение \eqref{eq:4} имеем следующее неравенство:
\[
\begin{aligned}
P\bigr[p\bigr(\left\{h\bigr(z\bigr)\not=f\bigr(z\bigr) \right\}\bigr)>\varepsilon\bigr] \leq |I|\sum_{j=1}^{k}{{d}\choose{j}}\left(1-\varepsilon\right)^{m-j},
\end{aligned}
\]
что и доказывает исходную теорему.
\end{proof}

\begin{thebibliography}{99}
\bibitem{Floyd1993}
	\textit{Floyd, S., Warmuth, M.} (1995) Sample Compression, Learnability, and the Vapnik-Chervonenkis Dimension. // Machine Learning 21, 269–304.  \url{https://doi.org/10.1023/A:1022660318680}
	
\bibitem{Vyigin2012}
    \textit{В.В.Вьюгин} КОЛМОГОРОВСКАЯ СЛОЖНОСТЬ И АЛГОРИТМИЧЕСКАЯ СЛУЧАЙНОСТЬ (2012) // МФТИ. 
    
\bibitem{Shey2015}
    \textit{Shay Moran, Amir Yehudayoff} Sample Compression Schemes for VC Classes (2015) // \url{https://www.cs.bgu.ac.il/~adsmb182/wiki.files/meni-lecture.pdf}


\end{thebibliography}

\end{document}
